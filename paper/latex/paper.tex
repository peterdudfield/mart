\documentclass[10pt,a4paper]{article}
\usepackage[letterpaper,margin=0.75in]{geometry}
\usepackage[utf8]{inputenc}
\usepackage{mdwlist}
\usepackage[T1]{fontenc}
\usepackage{textcomp}
\usepackage{tgpagella}
\pagestyle{empty}
\setlength{\tabcolsep}{0em}

\usepackage[defaultfam,tabular,lining]{montserrat} %% Option 'defaultfam'
%% only if the base font of the document is to be sans serif
\usepackage[T1]{fontenc}
\renewcommand*\oldstylenums[1]{{\fontfamily{Montserrat-TOsF}\selectfont #1}}

\usepackage{graphicx}

\begin{document}
\vspace*{\fill}
\begin{center}
\includegraphics[width=\textwidth]{../Images/paper_4.png}
\end{center}
\vspace*{\fill}

\begin{center}
\tiny
The dragon curve is made by taking a long strip of paper, and folding it in half many times. Unfolding the paper gives the patterns below. The pattern was first described in 1967 by Martin Gardner.
\end{center}

\end{document}