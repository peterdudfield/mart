\documentclass[10pt,a4paper]{article}
\usepackage[letterpaper,margin=0.75in]{geometry}
\usepackage[utf8]{inputenc}
\usepackage{mdwlist}
\usepackage[T1]{fontenc}
\usepackage{textcomp}
\usepackage{tgpagella}
\pagestyle{empty}
\setlength{\tabcolsep}{0em}

\usepackage[defaultfam,tabular,lining]{montserrat} %% Option 'defaultfam'
%% only if the base font of the document is to be sans serif
\usepackage[T1]{fontenc}
\renewcommand*\oldstylenums[1]{{\fontfamily{Montserrat-TOsF}\selectfont #1}}


\usepackage{graphicx}

\begin{document}
\vspace*{\fill}
\begin{center}
\includegraphics[width=0.75\textwidth]{../Images/mandelbrot_4_1000.png}
\end{center}
\vspace*{\fill}

\begin{center}
\tiny
This image contains four visualisations of the Mandelbrot set. The Mandelbrot set is a set complex numbers for which $f_c(z) = z^2 + c$ does not diverge. The image is self similar and contains properties linked with chaos theory. The set was first defined and printing in 1978.
\end{center}

\end{document}